\documentclass[11pt]{article}
\usepackage{fullpage}
\usepackage{fancyhdr}
\usepackage{epsfig}
\usepackage{multienum}
\usepackage{algorithm}
\usepackage[noend]{algorithmic}
\usepackage{amsmath,amssymb,amsthm}
\usepackage{ifsym}

\usepackage{cite}
\newtheorem{lemma}{Lemma}
\newtheorem*{lem}{Lemma}
\newtheorem{definition}{Definition}
\newtheorem*{definition*}{Definition}
\newtheorem{notation}{Notation}
\newtheorem*{claim}{Claim}
\newtheorem*{fclaim}{False Claim}
\newtheorem{observation}{Observation}
\newtheorem{conjecture}[lemma]{Conjecture}
\newtheorem{theorem}[lemma]{Theorem}
\newtheorem{corollary}[lemma]{Corollary}
\newtheorem{proposition}[lemma]{Proposition}
\newtheorem*{rt}{Running Time}

\def\P{\ensuremath{\mathcal{P}}}
\def\s{\ensuremath{{\bf s}}}
\def\p{\ensuremath{{\bf p}}}
\def\opt{\ensuremath{\textsc{opt}}}

\textheight=8.6in
\setlength{\textwidth}{6.44in}
\addtolength{\headheight}{\baselineskip} 
% enumerate uses a., b., c., ...
\renewcommand{\labelenumi}{\bf \alph{enumi}.}
% Sets the style for fancy pages (namely, all but first page)
\pagestyle{fancy}
\fancyhf{}
\renewcommand{\headrulewidth}{0.0pt}
\renewcommand{\footrulewidth}{0.4pt}
% Changes style of plain pages (namely, the first page)
\fancypagestyle{plain}{
  \fancyhf{}
  \renewcommand\headrulewidth{0pt}
  \renewcommand\footrulewidth{0.4pt}
  \renewcommand{\headrule}{}
}

\title{Screeps Lite: An Environment of Exploration, Emergence, Greediness, and Sabotage}
\author{Eren Guendelsberger, Jake Israel, and James Capuder}
\date{Spring 2017}

\begin{document}
\maketitle
\thispagestyle{plain}

\begin{abstract}
Q-Learning, and the numerous extensions thereof, has become a ubiquitous reinforcement learning technique. To explore the effectivness of Q-Learning in training multiple agents to farm resources, a training environment was implemented Python (3.5). The modular nature of the developed environment allowed for the simulation of various scenarious and reward structures, and for the possibility of future extensions.
\end{abstract}

\section{Introduction}

\section{Prior Research}

As discussed above, one shortcoming of \textit{Screeps} for this project was the lack of any prior Artificial Intelligence research reltaed to the game (perhaps foreshadowing the conclusions that lead to our shift in focus). Reinforcement Learning as a technique for building bots to play games, however, is a common research subject. Though an our environment and learning methods were relatively simple, resources providing background on Reinforcement Learning offered valuable supplemental information to that covered in lecture.

\section{Implementation}


\section{Results}

\section{Future Work}

\section{Conclusion}

\section{References}

\bibliography{report}

\end{document}

